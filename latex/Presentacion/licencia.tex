% Este diseño se corresponde con la licencia CC-BY-NC-SA.
% Por supuesto, puedes poner la licencia que mejor se adapte al propósito de tu trabajo.
% Recuerda que, si no se especifica ninguna licencia, esta -como cualquier creación artística- pasaría a estar licenciada con todos los derechos reservados (copyright).

\vspace{5cm}

\begin{flushright}

\begin{figure}
\includegraphics[width=0.10\textwidth,right]{figs/by-nc-sa.png}
\end{figure}

\vspace{0.2cm}

{\tiny 
(CC) \textbf{Vidal Pérez Bohoyo}\\ % TODO: pon aquí tu nombre cuando hagas el documento
\vspace{0.5cm}
\emph{
Este trabajo se entrega bajo licencia \href{https://creativecommons.org/licenses/by-nc-sa/3.0/es/}{CC BY-NC-SA}. \\
Usted es libre de \textit{(a) compartir}: copiar y redistribuir el material en \\
cualquier medio o formato; y \textit{(b) adaptar}: remezclar, transformar \\
y crear a partir del material. El licenciador no puede revocar estas \\
libertades mientras cumpla con los términos de la licencia. \\}
}

\end{flushright}

