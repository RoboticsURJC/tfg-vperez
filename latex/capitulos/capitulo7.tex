\chapter{Conclusiones}
\label{cap:capitulo7}

\begin{flushright}
\begin{minipage}[]{10cm}
\emph{Quizás algún fragmento de libro inspirador...}\\
\end{minipage}\\

Autor, \textit{Título}\\
\end{flushright}

\vspace{1cm}

Escribe aquí un párrafo explicando brevemente lo que vas a contar en este capítulo, que básicamente será una recapitulación de los problemas que has abordado, las soluciones que has prouesto, así como los experimentos llevados a cabo para validarlos. Y con esto, cierras la memoria.

\section{Conclusiones}

Enumera los objetivos y cómo los has cumplido.\\

Enumera también los requisitos implícitos en la consecución de esos objetivos, y cómo se han satisfecho.\\

No olvides dedicar un par de párrafos para hacer un balance global de qué has conseguido, y por qué es un avance respecto a lo que tenías inicialmente. Haz mención expresa de alguna limitación o peculiaridad de tu sistema y por qué es así. Y también, qué has aprendido desarrollando este trabajo.\\

Por último, añade otro par de párrafos de líneas futuras; esto es, cómo se puede continuar tu trabajo para abarcar una solución más amplia, o qué otras ramas de la investigación podrían seguirse partiendo de este trabajo, o cómo se podría mejorar para conseguir una aplicación real de este desarrollo (si es que no se ha llegado a conseguir).

\section{Corrector ortográfico}

Una vez tengas todo, no olvides pasar el corrector ortográfico de \LaTeX a todos tus ficheros \textit{.tex}. En \texttt{Windows}, el propio editor \texttt{TeXworks} incluye el corrector. En \texttt{Linux}, usa \texttt{aspell} ejecutando el siguiente comando en tu terminal:

\begin{verbatim}
aspell --lang=es --mode=tex check capitulo1.tex
\end{verbatim}
