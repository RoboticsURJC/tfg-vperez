\chapter{Objetivos}
\label{cap:capitulo3}

\begin{flushright}
\begin{minipage}[]{10cm}
\emph{Un objetivo sin un plan es solo un deseo}\\
\end{minipage}\\

Antoine de Saint-Exupéry\\
\end{flushright}

\vspace{1cm}

Tras haber enmarcado el contexto en el cual se encuentra este trabajo de fin de grado, se procede a realizar
una descripción del problema, requisitos, metodología y plan de trabajo usado.
\section{Descripción del problema}
\label{sec:descripcion}
El objetivo principal de este trabajo de fin de grado es desarrollar un brazo robótico desde cero. Se pretende que
el robot pueda ser empleado en la asignatura de Robótica Industrial en la Universidad Rey Juan Carlos para uso. Es por esto,
que se busca que el sistema creado sea asequible y fácilmente replicable mediante el uso de la impresión 3D.\\
Con el fin de alcanzar esta meta, se ha dividido el proyecto en los siguientes subobjetivos:

\begin{enumerate}
    \item Realizar una investigación acerca de los robots que actualmente están disponibles y que cumplan con 
          las características y objetivos deseados. Estos robots deberán tener un tamaño y costo similar, y 
          preferiblemente haber sido impresos en 3D. Se dará mayor relevancia a aquellos que utilicen un software y hardware 
          libres.
    \item Explorar diversas opciones de diseño para determinar la forma ideal del robot. Se analizará cuidadosamente 
          qué tipo de robot se ajusta mejor al uso que se le pretende dar, así como los grados de libertad necesarios. 
            
    \item Realizar una investigación exhaustiva sobre los componentes de hardware disponibles en el mercado, con el fin de seleccionar 
          aquellos que mejor se adapten a las necesidades y objetivos de construir un robot eficiente y funcional. Se llevará 
          a cabo un análisis detallado de los precios y características de cada componente, para seleccionar aquellos 
          que ofrezcan la mejor relación calidad-precio. Una vez evaluados todos los aspectos, se elegirá un conjunto 
          final de componentes para construir el robot de manera efectiva. 
    
    \item Diseño CAD (diseño asistido por ordenador) del brazo. Se pretende hacer uso de FreeCad, entre otras herramientas, para
          diseñar en 3D cada pieza que constituirá el robot. Puesto a que debe ser imprimible en una impresora FDM convencional,
          debe estar pensado para no necesitar soportes y necesitar de poco material.
    \item Emplear una impresora 3D convencional para materializar los diseños realizados anteriormente.
    \item Programar el software necesario para poder controlar el robot desde el ordenador.
    \item ROS de 

  
\end{enumerate}\


\section{Requisitos}
\label{sec:requisitos}
De este trabajo se espera un brazo robot de tipo industrial de bajo coste cuya fabricación completa esté por debajo de 200\euro.
Otro requisito es que la mayoría de las partes que componen el robot, sean imprimibles en cualquier impresora 3D de gama baja.\\\\
Además, se busca un consumo reducido para poder ser usado en robots móviles sin necesidad de estar conectado a una gran fuente de 
energía.\\\\
En cuanto a sus dimensiones, se busca un tamaño idóneo para su uso en escritorios y fácilmente portable. Esto implica que no necesariamente 
tiene que estar fijo al suelo, permitiendo su fácil traslado.\\\\
Se busca continuidad en el proyecto a largo plazo, por lo que una integración en ecosistema ROS 2 es una pieza clave. 

\section{Metodología}
\label{sec:metodologia}

Durante el desarrollo del trabajo se ha establecido un protocolo de reuniones semanales con el tutor a 
través de la plataforma Teams, con el objetivo de compartir los avances realizados y recibir retroalimentación 
sobre el trabajo. Además, cada semana se han propuesto las actividades a realizar, asegurando así una adecuada 
planificación y coordinación del proyecto. \\
Para el desarrollo del sistema se ha utilizado un repositorio en la plataforma GitHub\footnote{\url{https://github.com/RoboticsURJC/tfg-vperez}}, en el cual se ha ido
subiendo el código y diseños generados a lo largo del proyecto. Adicionalmente, en este mismo repositorio 
se ha incluido una Wiki\footnote{\url{https://github.com/RoboticsURJC/tfg-vperez/wiki}} con las explicaciones detalladas de todas las actividades llevadas a cabo durante estos 
meses de trabajo. De esta manera, se ha creado un registro completo y accesible de todo el proceso de desarrollo del sistema.

\section{Plan de trabajo}
\label{sec:plantrabajo}
El desarrollo del TFG ha estado dividido en dos etapas. La primera, comenzó en octubre y fue abandonada en enero. 