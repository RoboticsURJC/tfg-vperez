\chapter{Objetivos}
\label{cap:capitulo3}

\begin{flushright}
\begin{minipage}[]{10cm}
\emph{Un objetivo sin un plan es solo un deseo.}\\
\end{minipage}\\

Antoine de Saint-Exupéry\\
\end{flushright}

\vspace{1cm}

\noindent Tras haber enmarcado el contexto en el cual se encuentra este trabajo de fin de grado, se procede a realizar
una descripción del problema, requisitos, metodología y plan de trabajo usado.
\section{Descripción del problema}
\label{sec:descripcion}
\noindent Este trabajo de fin de grado nace de la necesidad de abordar el problema existente en las universidades, donde la escasez de 
robots disponibles, y la dificultad para acceder a ellos, limitan la experiencia práctica en robótica.

Este problema es causado en gran medida por elevado coste y fragilidad de los robots utilizados, limitando la interacción plena de 
los estudiantes por temor a dañarlos. Además, la cantidad limitada de ellos en relación con el número de carreras que 
quieren utilizarlos, limita su disponibilidad. Sumado a ello, los profesores tienen que hacer frente a la burocracia asociada a su uso.
\\ 
\indent La solución propuesta busca superar estas limitaciones, proporcionando un robot casero impreso en 3D, que es
económico, accesible y útil en el proceso de aprendizaje práctico de los estudiantes. Concretamente, el objetivo 
principal de este trabajo de fin de grado es desarrollar un brazo robótico que pueda ser empleado en la 
asignatura de Robótica Industrial de esta universidad. Asimismo, se busca que el sistema creado sea barato y fácilmente replicable, para que 
se pueda disponer de un número elevado de estos en el aula.
\newpage
Con el fin de alcanzar este objetivo, se consideran los siguientes subobjetivos:

\begin{enumerate}
    \item Realizar una investigación acerca de los robots que actualmente están disponibles y que cumplan con 
          las características y objetivos deseados. 
    \item Explorar diversas opciones de diseño para determinar la forma del robot. 
      
    \item Realizar una investigación exhaustiva sobre los componentes de hardware disponibles en el mercado, con el fin de seleccionar 
          aquellos que mejor se adapten a las necesidades y objetivos.
    
    \item Realizar el diseño \acs{CAD} de las piezas mediante el uso de herramientas libres. 

    \item Emplear una impresora 3D convencional para materializar las piezas realizados.

    \item Realizar el software necesario para poder controlar el robot desde el ordenador.

    \item Integrar el robot en el ecosistema \ac{ROS}2/MoveIt. 
 
\end{enumerate}\

\section{Requisitos}
\label{sec:requisitos}
\noindent Con el fin de solucionar el problema descrito, se han establecido los siguientes requisitos:
\begin{enumerate}
      \item La fabricación del brazo robótico no debe suponer un coste superior a 200\euro.
      \item La mayoría de las partes que componen el robot deben ser imprimibles en cualquier impresora 3D convencional.
      \item A fin de poder garantizar la portabilidad del robot, este debe tener un consumo inferior a 25 vatios.
      \item En cuanto a sus dimensiones, se busca un tamaño idóneo para su uso sobre un escritorio. Esto implica que no necesariamente 
      tiene que estar unido al suelo, permitiendo así su fácil traslado.
      \item Es necesario que sea simple de montar y esté compuesto del menor número de piezas posibles, con el fin de poder crear varias unidades 
      en poco tiempo. 
      \item Se busca continuidad en el proyecto a largo plazo, por lo que debe tener integración con el ecosistema \acs{ROS} 2. 

\end{enumerate}\

\section{Metodología}
\label{sec:metodologia}
\noindent Durante el desarrollo del trabajo se ha establecido un protocolo de reuniones semanales con el tutor a 
través de la plataforma Teams, con el objetivo de compartir los avances realizados y recibir retroalimentación 
sobre el trabajo. Además, cada semana se han propuesto las actividades a realizar, asegurando así una adecuada 
planificación y coordinación del proyecto. \\
\indent Para el desarrollo del sistema se ha utilizado un repositorio en la plataforma GitHub\footnote{\url{https://github.com/RoboticsURJC/tfg-vperez}}, en el cual se ha ido
subiendo el código y diseños generados a lo largo del proyecto. Adicionalmente, en este mismo repositorio 
se ha incluido una Wiki\footnote{\url{https://github.com/RoboticsURJC/tfg-vperez/wiki}} con las explicaciones detalladas de todas las actividades llevadas a cabo durante estos 
meses de trabajo. De esta manera, se ha creado un registro completo y accesible de todo el proceso de desarrollo del sistema.

\section{Plan de trabajo}
\label{sec:plantrabajo}
\noindent El desarrollo del TFG ha estado dividido en las siguientes etapas:

\begin{enumerate}
\item \textit{Investigación del estado del arte}. Fase inicial en la que se realizaron diferentes
búsquedas en plataformas online como Google Schoolar con el fin de encontrar posibles soluciones al problema descrito. Se buscaban 
palabras clave como ``DIY'', ``educational'', ``low-cost'', ``robot'', ``arm'', ``DOF'' y ``3D printed''. Además se priorizó aquellos trabajos que utilizaban  
software y hardware libre. 
\item \textit{Encontrar la forma del robot ideal}. Se analizó cuidadosamente qué tipo de robot se ajusta mejor a los requisitos propuestos. Se establecieron los 
grados de libertad necesarios y su principio de funcionamiento. Para ello, se investigó los posibles tipos de articulaciones y 
su configuración.

\item \textit{Análisis del mercado de componentes}. En esta fase, se realizó un análisis detallado de los precios y características 
técnicas de cada componente, para seleccionar aquellos que ofrezcan la mejor relación calidad-precio. Una vez evaluados todos los 
aspectos, se eligió el conjunto final de componentes que componen el robot.
\newpage
\item \textit{Diseño CAD}. Tras haber definido el concepto y haber seleccionado los componentes físicos que lo componen, se comenzó con 
el diseño del manipulador. Primero de todo, se realizaron numerosos bocetos a mano alzada para establecer la posición espacial de cada 
pieza. Posteriormente se concretó la forma exacta de cada pieza teniendo en cuenta que dichas piezas debían ser impresas en 3D y requerir 
de la menor cantidad de material posible. Finalmente, se hizo uso de la herramienta \nameref{subsec:freecad} para diseñar por ordenador 
todas y cada una de las piezas.

\item \textit{Impresión 3D}. En esta fase del proyecto, se materializaron los diseños previos mediante la técnica de impresión 3D haciendo 
uso de una impresora de filamento (tipo \acs{FDM}) convencional. Previamente, se habían realizado pruebas de tolerancias para garantizar 
que las piezas impresas cumplían con las medidas especificadas en el diseño y calibrar la dilatación térmica en función de eso. Esta fase 
finalizó con el ensamblado de todos los componentes y piezas.

\item \textit{Desarrollo del software de control}. Una vez fue construido el robot, se desarrolló toda la programación de bajo nivel 
necesaria para poder controlar el robot desde el ordenador. Para ello se investigó las posibles opciones de comunicación con la electrónica elegida. Además,  
se calculó todas la matemáticas necesarias para su control numérico.

\item \textit{Integración en ROS2 y MoveIt}. Tras poder controlar el robot mediante el ordenador, se realizó una integración en el 
ecosistema ROS. Se describió el robot mediante el formato URDF y se integraron los controladores de articulaciones de ROS necesarios 
para su control mediante \textit{topics}. Posteriormente, se realizó la integración con el \textit{framework} MoveIt a partir del paquete 
de descripción generado previamente. 

\item \textit{Evaluación del desempeño}. En esta etapa final, se evaluó las distintas capacidades técnicas del manipulador con el 
software final. Se encontraron las limitaciones físicas del robot y se generaron una serie de tablas con los resultados obtenidos.

\end{enumerate}


