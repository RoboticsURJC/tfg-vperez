\chapter{Plataforma de desarrollo}
\label{cap:capitulo3}

\begin{flushright}
\begin{minipage}[]{10cm}
\emph{Quizás algún fragmento de libro inspirador...}\\
\end{minipage}\\

Autor, \textit{Título}\\
\end{flushright}

\vspace{1cm}

\section{Python}
\label{sec:pyhton}

Escribe aquí un párrafo explicando brevemente lo que vas a contar en este capítulo. En este capítulo, explica qué has usado a nivel hardware y software para poder desarrollar tu trabajo: librerías, sistemas operativos, plataformas, entornos de desarrollo, etc.

\section{Grbl}
\label{sec:grbl}
Grbl\footnote{\url{https://github.com/gnea/grbl}} es un firmware de código abierto usado para controlar máquinas llamadas CNC (Control Numérico por Computadora). Este firmware se ejecuta en 
un microcontrolador, que se encuentra dentro de la controladora de la máquina CNC, en nuestro caso, la placa base del robot. \\
Básicamente, convierte las instrucciones de código G, que posteriormente hablaremos de él, en señales eléctricas que se envían a los motores de la máquina. Además, 
comprueba los diferentes sensores de la máquina, como pueden ser los finales de carrera, para establecer los límites físicos de cada movimiento. \\
Grbl es flexible por lo que podemos cambiar la configuración para adaptarla a un caso de uso concreto. De hecho, aunque solo soporta movimientos lineales,
en este trabajo se abordará la configuración necesaria para para poder usar las articulaciones rotativas del nuestro robot.

\section{Código G}
\label{sec:gcode}
Se trata 

\section{MKS DLC32}
\label{sec:mksdlc32}
Se trata de una placa destinada al mundo de las máquinas de grabado láser. Ha sido creada por \textit{MakerBase} y es considerada 
\textit{Open Hardware} por lo que toda la información de la placa puede encontrarse en su repositorio de Github\footnote{\url{https://github.com/makerbase-mks/MKS-DLC32}}.
Es fácilmente adquirible por \textit{Aliexpress} por un precio que ronda los 16\euro. Está basada en el microcontrolador de 32 bits: ESP32. 
Se trata de un dispositivo muy asentado en la comunidad \textit{maker} debido a su bajo coste e integración en el 
ecosistema Arduino. De hecho, gracias a su conectividad wifi y bluetooth ha ganado terreno a los microcontroladores Atmega que incorporan los propios Arduinos.\\
Esta placa la ideal para este proyecto debido a que cuenta con la posibilidad de controlar 
hasta 3 motores y es completamente compatible con Grbl. Además dispone de una salida de potencia regulable controlable mediante Grbl que nos 
permite alimentar dispositivos. Estos podrían ser: electroimán (tipo de imán que es activado mediante electricidad), motor CC (motor de corriente 
contínua) entre otros. Además se puede aprovechar las salidas PWM (\textit{Pulse Width Modulation}) para conectar un grabador láser o un servo. 
El rango de funcionamiento es de 12 a 24 voltios por lo que es adecuado para ser alimentado mediante baterías y con cargadores de ordenador. 

\section{Motores paso a paso}
\label{sec:motores}
Un motor paso a paso es un tipo de motor que se mueve en pequeños pasos o incrementos discretos en lugar de girar continuamente. Estos pasos 
son controlados por señales eléctricas que hacen girar al motor una cantidad específica de grados cada vez que se envía una señal. Debido a que se tiene 
control sobre su avance son una excelente opción si se quiere tener un motor que sea capaz de posicionarse en un ángulo concreto con exactitud. A pesar
de su gran precisión, los motores paso a paso convencionales no tienen el conocimiento absoluto de su posición, por lo que todos los movimientos son 
relativos. Esto hace que se requiera de un \textit{homing} (proceso en el cual la máquina CNC lleva las partes móviles a una 
posición conocida) en el arranque de la máquina para conocer su estado antes de operar.

\section{Controladores de motores}
\label{sec:controladorPAP}
Un controlador paso a paso es el módulo hardware capaz de trasformar las señales lógicas que le envía el controlador en una serie de pulsos de
potencia que excitarán las bobinas del motor en un cierto orden para lograr el movimiento. 
Existen motores bipolares, unipolares e híbridos. La diferencia entre ellos radica en la disposición de las bobinas de su interior. Los más usados 
en impresoras 3D y CNC son los bipolares. Son reconocibles debido a que tienen 4 cables.
En este tipo de placas base se pueden instalar distintos modelos de controladores. Cada uno de ellos tiene unas prestaciones diferentes y por tanto 
un precio distinto. Unos ofrecen mayor capacidad de corriente (para controlar motores más grandes), pulsos más suaves que reducen el 
ruido sonoro y las vibraciones, medición en tiempo real de la corriente consumida para conocer el final de una articulación, entre otras tecnologías.  
