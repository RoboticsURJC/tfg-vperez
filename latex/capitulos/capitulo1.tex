\chapter{Introducción}
\label{cap:capitulo1}
\setcounter{page}{1}

\begin{flushright}
\begin{minipage}[]{10cm}
\emph{El éxito es la capacidad de ir de fracaso en fracaso sin perder el entusiasmo.}\\
\end{minipage}\\

Winston Churchill\\
\end{flushright}

\vspace{1cm}


\section{Robótica}
\label{sec:rob}

donde iene este termino
de las esntanforms y primero robot

Los robots se pueden clasificar en dos grupos.
\subsection{Robots de servicio}
Un robot de servicio es un tipo de robot diseñado para realizar tareas en beneficio de los seres humanos. Estos robots están 
destinados a interactuar directamente con las personas y ayudar en diversas actividades. 
Debido a la necesidad de interactuar con los humanos, están equipados con gran variedad de sensores, actuadores y sistemas de 
inteligencia artificial que les permiten percibir y comprender el entorno que los rodea. 
En función de su ámbito de uso, pueden llegar a realizar una amplia gama de tareas, como limpieza y mantenimiento del hogar, 
asistencia en la intervención médica, entrega de alimentos y productos, cuidado de personas mayores, entre otros.
\subsubsection{Robots en rescates}
\subsubsection{Robots en el espacio}
\subsubsection{Robots en medicina}

\subsection{Robots industriales}
Se entiende por robot industrial a una máquina automatizada diseñada específicamente para llevar a cabo tareas en entornos industriales. 
Disponen de numerosas articulaciones y una gran capacidad de maniobrabilidad. Su objetivo principal es remplazar a un 
operario humano en tareas aburridas, sucias, peligrosas y exigentes (\textit{4D: Dull, Dirty, Dangerous and Demanding}).

\section{Robótica industrial}
\label{sec:rob_industrial}


\subsection{SCARA}
\subsection{Articulados}
\subsection{Paralelos}
\subsection{Cartesianos}

\section{Robótica educativa}
\label{sec:rob_educativa}
\subsection{Robótica en institutos}
\subsection{Robótica en universidades}

\section{Robótica de bajo coste}
\label{sec:rob_bajo:coste}

