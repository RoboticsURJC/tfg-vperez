\chapter{Introducción}
\label{cap:capitulo1}
\setcounter{page}{1}

\begin{flushright}
\begin{minipage}[]{10cm}
\emph{El éxito es la capacidad de ir de fracaso en fracaso sin perder el entusiasmo.}\\
\end{minipage}\\

Winston Churchill\\
\end{flushright}

\vspace{1cm}

La tendencia de la industra hacia la automatización total ha ido en aumento en los últimos años, 
haciendo que la demanda de robots industriales se dispare en todo el mundo. Un ejemplo de ello, son 
las megafactorías en el sector del automóvil. Se tratan de inmensas fábricas con un componente humano
mínimo y con una automatización cada día mayor, mediante el uso de la robótica industrial. \\No solo se hace
uso de grandes robots pesados, sino también de flotas de pequeños pero ágiles robots que desempeñan tareas
en cintas trasportadoras, ensamblado de placas base\ldots  Debido al aumento en su uso y del progreso 
en estas tecnologías, es necesario formar cada vez más ingenieros que ejercerán en este campo. Es por esto que se 
necesitan herramientas adaptadas a las nuevas tecnologías y fácilmente accesibles para estudiantes y centros.    \\

\section{Robótica industrial}
\label{sec:miseccion} % etiqueta para luego referenciar esta sección

La robótica industrial es la rama de la ingeniería dedicada al diseño, construcción, operación y mantenimiento 
de robots utilizados en la automatización de procesos industriales. Estos robots pueden realizar una amplia variedad 
de tareas, desde la manipulación y transporte de materiales, hasta el ensamblaje y soldadura de piezas. Entre las aplicaciones 
más importantes encontramos las siguientes:

\begin{itemize}
  \item \textit{Automatización de líneas de ensamblaje.} Esta tecnología se encuentra ampliamente asentada en las líneas de  producción 
                                                        del sector del automotriz. Un ejemplo es la 
  \item \textit{Soldadura.} Es utilizada en soldadura de estructuras mecánicas debido a su alta precisión y capacidad de realizar 
                            la misma soldadura perfecta una y otra vez.
  \item \textit{Investigación y desarrollo en laboratorios.} Los brazos robóticos realizan tareas de laboratorio repetitivas y precisas, 
                              lo que puede acelerar el proceso de investigación y desarrollo de nuevos productos médicos.
 \end{itemize}\

\section{Robótica en educación}
\label{sec:segundaseccion}
La robótica en la educación es una disciplina que ha cobrado una gran relevancia en los últimos años debido a la creciente necesidad 
de formar a las nuevas generaciones en competencias tecnológicas. 
\\En las escuelas de secundaria, se ha convertido en una herramienta pedagógica eficaz para desarrollar habilidades y conocimientos en áreas como la programación, 
la matemática, la electrónica y la resolución de problemas. Esto es conocido como \textit{STEM} (Science, Technology, Engineering and Mathematics). Los 
estudiantes aprenden a diseñar, construir y programar robots simples para llevar a cabo una tarea específica, lo que les ayuda a comprender  
los conceptos de ciencia y tecnología de una manera más práctica e interactiva.\\
\begin{figure} [h!]
  \begin{center}
    \includegraphics[width=8cm]{figs/roomba}
  \end{center}
  \caption{Robots educativos en la escuela secundaria.}
  \label{fig:robSecundaria}
\end{figure}\
\\En el nivel universitario, la robótica se ha convertido en una disciplina esencial para formar a los futuros ingenieros. Los estudiantes aprenden 
a diseñar y construir robots más avanzados, y refuerzan sus habilidades de programación y control de sistemas complejos. Además se hacen uso de más tipos 
de robot, como pueden ser, industriales o plataformas robóticas móviles reales. Al ser sistemas usados en el mundo profesional, los estudiantes pueden 
aprender con el robot que usará en un futuro. Aunque es verdad que las universidades disponen de algunas unidades, no siempre son accesibles para el 
estudiante por diversas razones.
\newpage
\begin{figure} [h!]
  \begin{center}
    \includegraphics[width=8cm]{figs/roomba}
  \end{center}
  \caption{Robots educativos en universidades.}
  \label{fig:robUniversidades}
\end{figure}\
\\En resumen, la robótica en la educación es una herramienta poderosa para fomentar el aprendizaje y la innovación en las nuevas 
generaciones. Desde la escuela hasta la universidad, la robótica se ha convertido en una disciplina clave para formar a los futuros 
líderes tecnológicos del mundo.

No olvides incluir imágenes y referenciarlas, como la Figura \ref{fig:robUniversidades}.


\subsection{Números}
\label{sec:subseccion}

En lugar de tener secciones interminables, como la Sección \ref{sec:miseccion}, divídelas en subsecciones.

Para hablar de números, mételos en el entorno \textit{math} de \LaTeX, por ejemplo, $1.5Kg$. También puedes usar el símbolo del Euro como aquí: 1.500\euro.

\subsection{Listas}

Cuando describas una colección, usa \texttt{itemize} para ítems o \texttt{enumerate} para enumerados. Por ejemplo:

\begin{itemize}
 \item \textit{Entorno de simulación.} Hemos usado dos entornos de simulación: uno en 3D y otro en 2D.
 \item \textit{Entornos reales.} Dentro del campus, hemos realizado experimentos en Biblioteca y en el edificio de Gestión.
\end{itemize}\

\begin{enumerate}
 \item Primer elemento de la colección.
 \item Segundo elemento de la colección.
\end{enumerate}\

\paragraph{Referencias bibliográficas}
\label{sec:referencias}

Cita, sobre todo en este capítulo, referencias bibliográficas que respalden tu argumento. Para citarlas basta con poner la instrucción \verb|\cite| con el identificador de la cita. Por ejemplo: libros como \cite{vega12e}, artículos como \cite{vega19b}, URLs como \cite{vega19a}, tesis como \cite{vega18b}, congresos como \cite{vega18a}, u otros trabajos fin de grado como \cite{vega08b}.

Las referencias, con todo su contenido, están recogidas en el fichero \texttt{bibliografia.bib}. El contenido de estas referencias está en formato \texttt{BibTex}. Este formato se puede obtener en muchas ocasiones directamente, desde plataformas como \texttt{Google Scholar} u otros repositorios de recursos científicos.

Existen numerosos estilos para reflejar una referencia bibliográfica. El estilo establecido por defecto en este documento es APA, que es uno de los estilos más comunes, pero lo puedes modificar en el archivo \texttt{memoria.tex}; concretamente, cambiando el campo \verb|apalike| a otro en la instrucción \verb|\bibliographystyle{apalike}|. 

\

\

\

Y, para terminar este capítulo, resume brevemente qué vas a contar en los siguientes.
