\chapter{Desarrollo hardware del manipulador}
\label{cap:capitulo5}

\vspace{1cm}

En este capítulo se aborda el desarrollo necesario para, a partir de un concepto, acabar construyendo un prototipo real funcional. Se 
hace incisión en cada etapa necesaria para este cometido.


Escribe aquí un párrafo explicando brevemente lo que vas a contar en este capítulo. En este capítulo (y quizás alguno más) es donde, por fin, describes detalladamente qué has hecho y qué experimentos has llevado a cabo para validar tus desarrollos.

\section{Concepto}
En esta sección se expone como ha sido el proceso de encontrar y definir la idea fundamental del robot, en función de 
los objetivos propuestos, evaluando las distintas opciones para encontrar el que mejor se adapte. Se trata de balance

Primero, se debe conocer el numero de grados de libertad que se ajuste a los requisitos establecidos en la sección \ref{sec:requisitos}. En base 
a los requisitos 1 y 5, que limitan en cuanto a precio de fabricación y complejidad de los mismos, se ha decidido limitar los grados de 
libertad a un máximo de 4. 

hablamos del tipo de joints que existente
Existen muchos tipos de joints entre los que destacan:

\begin{itemize}
\item Revolución: Este tipo de \textit{joint} permite el movimiento de rotación alrededor de un eje fijo. Está restringido a 1 DOF. Es comúnmente usado en todo tipo de robots, sobre todo en industriales.
\item Prismático: En este tipo de \textit{joint}, las partes del robot se pueden desplazar linealmente a lo largo de un eje específico. Está restringido a 1 DOF. Es usado ptincipalmente en 
los carros de avance de robot industriales y en máquinas CNC.
\item Esférico: Esta articulación permite la rotación en cualquier dirección. Está restringido a 3 DOF y suele ser usado como articulación pasiva, es decir, que 
que su movimiento depende de restricciones externas.
\item Cilíndrica: Esta articulación permite el movimiento de rotación alrededor de un eje y también un desplazamiento lineal a lo largo del mismo. Combina los movimientos rotacionales y prismáticos. Está restringido a 2 DOF. Es usado en el extremo de los 
robots tipo \textit{Scara}.
\item Planar: Este tipo de joint permite el movimiento en un plano específico. Restringe el movimiento a 2 grados de libertad para el posicionamiento, y otro 
para la rotación en el eje perpendicular al plano. No es muy utilizado ya que es complejo de implementar.
\end{itemize}


Con hasta 4 grados de libertad podemos realizar los siguientes robots:
Tipos:

\begin{itemize}
\item \ac{SCARA}: son una categoría de robots industriales ampliamente utilizados en aplicaciones de ensamblaje electrónico, operaciones de 
\textit{pick and place}(coger componentes y situarlos en una determinada posición) y empaquetado de productos, entre otras.
Se caracterizan por su diseño de brazo articulado y su capacidad para realizar movimientos rápidos 
y precisos en un plano horizontal. Este tipo de robot suele tener tres o cuatro grados de libertad, en función de si en el extremo del 
robot puede rotar sobre si mismo o no. 

\begin{figure} [h!]
  \centering    
  \subfigure[Ensamblado de baterías]{\label{fig:fanuc}\includegraphics[width=0.3\linewidth ]{figs/fanuc.jpg}}
  \hspace{3cm}
  \subfigure[Espacio de trabajo]{\label{fig:fanuc_ws}\includegraphics[width=0.3\linewidth]{figs/fanuc_spacio.jpg}}
  \caption[Fanuc]{Ejemplo: \href{https://www.fanuc.eu/uk/en/robots/robot-filter-page/scara-series/scara-sr-3ia}{Fanuc SR-3iA}}
\end{figure}

\item \ac{RR convencional}:
\begin{figure} [h!]
  \centering    
  \subfigure[Robot real]{\label{fig:rebel}\includegraphics[width=0.3\linewidth ]{figs/rebel4dof.jpg}}
  \hspace{3cm}
  \subfigure[Espacio de trabajo]{\label{fig:rebel_ws}\includegraphics[width=0.4\linewidth]{figs/igus_space.png}}
  \caption[Rebel]{Ejemplo: \href{https://www.igus.es/product/20962?artNr=REBEL-4DOF-01}{REBEL-4DOF-01}}
\end{figure}

\end{itemize}


Basado en paralelos

\section{Modelo alámbrico}
\subsection{En qué consiste}
\label{subsec:eqc_mod_alambrico}
El modelo alámbrico es una forma de analizar el movimiento de un sistema mecánico compuesto por ejes y eslabones. Este 
enfoque simplifica la representación visual al destacar las relaciones espaciales entre las diferentes partes del sistema mediante 
líneas y conexiones simbólicas, en lugar de mostrar detalles realistas del manipulador. 
\begin{figure} [h!]
  \begin{center}
    \includegraphics[width=15cm]{figs/pinza_evol.png}
  \end{center}
  \caption{Pinza paralela con 1 grado de libertad}
  \label{fig:mod_pinza_figure}
\end{figure}\ 

\subsection{Modelo alámbrico del brazo MeArm \ref{fig:mearm}}
\label{subsec:mod_mearm}


\section{Dinámica}

\section{DH}

\section{Bocetos}

\section{Diseño CAD}

\section{Diseño CAD}
