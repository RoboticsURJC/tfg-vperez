\chapter{Objetivos}
\label{cap:capitulo2}

\begin{flushright}
\begin{minipage}[]{10cm}
\emph{Un objetivo sin un plan es solo un deseo}\\
\end{minipage}\\

Antoine de Saint-Exupéry\\
\end{flushright}

\vspace{1cm}

Tras haber enmarcado el contexto en el cual se encuentra este trabajo de fin de grado, se procede a realizar
una descripción del problema, requisitos, metodología y plan de trabajo usado.
\section{Descripción del problema}
\label{sec:descripcion}
El objetivo principal de este trabajo de fin de grado es desarrollar un brazo robótico desde cero. Se pretende que
el robot pueda ser empleado en la asignatura de Robótica Industrial en la Universidad Rey Juan Carlos para uso. Es por esto,
que se busca que el sistema creado sea asequible y fácilmente replicable mediante el uso de la impresión 3D.\\
Con el fin de alcanzar esta meta, se ha dividido el proyecto en los siguientes subobjetivos:

\begin{enumerate}
    \item Realizar una investigación acerca de los robots que actualmente están disponibles y que cumplan con 
          las características y objetivos deseados. Estos robots deberán tener un tamaño y costo similar, y 
          preferiblemente haber sido impresos en 3D. Se dará mayor relevancia a aquellos que utilicen un software y hardware 
          libres.
    \item Inverstigar la forma y dinamica, grados de libertad que se quieren

    \item Realizar una investigación exhaustiva sobre los componentes de hardware disponibles en el mercado, con el fin de seleccionar 
          aquellos que mejor se adapten a las necesidades y objetivos de construir un robot eficiente y funcional. Se llevará 
          a cabo un análisis detallado de los precios y características de cada componente, para seleccionar aquellos 
          que ofrezcan la mejor relación calidad-precio. Una vez evaluados todos los aspectos, se elegirá un conjunto 
          final de componentes para construir el robot de manera efectiva. 
    
    \item Diseñar el modelo en 3d pensando en que debe ser imprimible sin soportes y facilmente en impresoras baratas
    \item Montar el robot y hacer correcciones
    \item Crear el software necesario
    \item ROS de 

  
\end{enumerate}\


\section{Requisitos}
\label{sec:requisitos}
De este trabajo se espera un sistema robótico de bajo coste cuya fabricación completa, incluyendo hasta la última tuerca, esté 
por debajo de 200€.\\\\
Otro requisito es que la mayoría de las partes que componen el robot, sean imprimibles incluso en una impresora 3D de gama baja. Esto 
hace que cualquier persona pueda construirse el suyo sin necesitar de muchas herramientas.\\\\
Además, se busca un consumo reducido para poder ser usado en robots móviles sin necesidad de estar conectado a una gran fuente de 
energía.\\\\
En cuanto a sus dimensiones, se busca un tamaño idóneo para su uso en escritorios y fácilmente portable. Esto implica que no necesariamente 
tiene que estar fijo al suelo, permitiendo su fácil traslado.\\\\
Se busca continuidad en el proyecto a largo plazo por lo que una integración en ecosistema ROS 2 es una pieza clave. 

\section{Metodología}
\label{sec:metodologia}

Qué paradigma de desarrollo software has seguido para alcanzar tus objetivos.

\section{Plan de trabajo}
\label{sec:plantrabajo}

Qué agenda has seguido. Si has ido manteniendo reuniones semanales, cumplimentando objetivos parciales, si has ido afinando poco a poco un producto final completo, etc.
