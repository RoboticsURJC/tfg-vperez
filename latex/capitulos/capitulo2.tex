\chapter{Estado del arte}
\label{cap:capitulo2}
En esta sección, se expondrá el estado del arte de los robots industriales en educación. Esta fase fundamental de la 
investigación se ha completado gracias a la búsqueda en diversas fuentes de renombre, con el fin de 
recopilar información que pueda ser de utilidad para desarrollar el presente proyecto de fin de grado. Como resultado, se 
han seleccionado una serie de trabajos relevantes y significativos en la materia, que se procederán a analizar a continuación.
\begin{itemize}
    \item En \cite{KRIMPENIS2020103} se presenta una solución industrial barata para crear un robot capaz de hacer operaciones de mecanizado (fabricación 
    de piezas mediante operaciones de corte). Es llamado Hydra y esta dotado de 6 \ac{DOF}. Tiene un alcace máximo de casi un metro y un 
    peso que ronda los 13 Kg. Pese a todas estas funciones, está fabricado mediante impresión 3D. 
    Además tiempo después, se publicó la segunda parte de este artículo: \cite{PAPAPASCHOS2020109}. En el cual se aborda el diseño software y de control 
    que se ha desarrollado para controlar dicho brazo. \\
    En base a lo descrito en estos artículos se han obtenido los siguientes puntos fuertes del proyecto:
    \begin{itemize}
        \item Tiene una repitividad de ±0.04 mm.
        \item Tiene un espacio de trabajo muy amplio.
        \item Tener más de 3 grados de libertad le permiten alcanzar gran cantidad de puntos con distintas orientaciones.
        \item Según el árticulo, tiene una capacidad de carga máxima de 12 kilogramos. A pesar de esto, se comenta que en la práctica el peso en el 
        extremo del robot debe ser menor a 5 Kg, por lo que su carga útil rondará los 3 Kg para un funcionamiento acceptable. Esto lo sitúa a la par 
        del robot comercial ABB IRB 120\footnote{\url{https://new.abb.com/products/es/3HAC031431-001/irb-120}}  
    \end{itemize}\
    En cambio, tambíen se deben mencionar los siguientes puntos débiles:
    \begin{itemize}
        \item Carece de integración en ROS
        \item Su coste en materiales supera los 1000\euro  por lo que no es lo suficientemente asequible para su uso académico.
        \item Según el artículo: "Total design, manufacturing and assembly time was about 200h", lo que implica que es costoso en tiempo 
        crear varias unidades y requiere de cierta habiliadad para montarlo correctamente.
        \item En este artículo no se proporcionan los ficheros necesarios para poder replicacarlo. Además, se menciona que las piezas han sido 
        diseñadas mediante el software privativo SolidWorks\textsuperscript{\tiny\textregistered} por lo que lo hace más dificil y costoso de editar.
    \end{itemize}\
\end{itemize}\
\vspace{1cm}
