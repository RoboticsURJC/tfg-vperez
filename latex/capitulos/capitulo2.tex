\chapter{Objetivos}
\label{cap:capitulo2}

\begin{flushright}
\begin{minipage}[]{10cm}
\emph{Un objetivo sin un plan es solo un deseo}\\
\end{minipage}\\

Antoine de Saint-Exupéry\\
\end{flushright}

\vspace{1cm}

Tras haber enmarcado el contexto en el cual se encuentra este trabajo de fin de grado, se procede a realizar
una descripción del problema, requisitos, metodología y plan de trabajo usado.
\section{Descripción del problema}
\label{sec:descripcion}
El objetivo principal de este trabajo de fin de grado es crear y documentar el proceso de construir tu propio brazo robot 
desde cero. Además, se busca que el sistema creado sea asequible y fácilmente replicable mediante el uso de la impresión 3D. El campo 
de uso de este robot está en la robótica educativa en universidades.
Para lograr dicha meta, se ha dividido el problema en estos
subobjetivos:
\section{Requisitos}
\label{sec:requisitos}
De este trabajo se espera un sistema robótico de bajo coste cuya fabricación completa, incluyendo hasta la última tuerca, esté 
por debajo de 200€.\\\\
Otro requisito es que la mayoría de las partes que componen el robot, sean imprimibles incluso en una impresora 3D de gama baja. Esto 
hace que cualquier persona pueda construirse el suyo sin necesitar de muchas herramientas.\\\\
Además, se busca un consumo reducido para poder ser usado en robots móviles sin necesidad de estar conectado a una gran fuente de 
energía.\\\\
En cuanto a sus dimensiones, se busca un tamaño idóneo para su uso en escritorios y fácilmente portable. Esto implica que no necesariamente 
tiene que estar fijo al suelo, permitiendo su fácil traslado.\\\\
Se busca continuidad en el proyecto a largo plazo por lo que una integración en ecosistema ROS 2 es una pieza clave. 

\section{Metodología}
\label{sec:metodologia}

Qué paradigma de desarrollo software has seguido para alcanzar tus objetivos.

\section{Plan de trabajo}
\label{sec:plantrabajo}

Qué agenda has seguido. Si has ido manteniendo reuniones semanales, cumplimentando objetivos parciales, si has ido afinando poco a poco un producto final completo, etc.
