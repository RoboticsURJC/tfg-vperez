\chapter{Conclusiones}
\label{cap:capitulo7}

\begin{flushright}
\begin{minipage}[]{10cm}
\emph{Quizás algún fragmento de libro inspirador...}\\
\end{minipage}\\

Autor, \textit{Título}\\
\end{flushright}

\vspace{1cm}

\noindent En el último capítulo, se hará mención a los logros alcanzados en términos de objetivos y se presentarán 
las conclusiones obtenidas en este proyecto. También se discutirán las habilidades y conocimientos 
adquiridos durante su desarrollo, así como las posibles mejoras futuras a tener en cuenta.
\section{Objetivos cumplidos}
Primeramente cabe destacar que se ha logrado cumplir el objetivo principal de este trabajo:
Enumera los objetivos y cómo los has cumplido.\\

Enumera también los requisitos implícitos en la consecución de esos objetivos, y cómo se han satisfecho.\\
\section{Competencias adquiridas}
En el desarrollo de este trabajo se han adquirido una gran cantidad de conocimientos y competencias y conocimientos, 
entre los cuales destacan:
\begin{itemize}
\item Conocimientos avanzados acerca de Grbl y su funcionamiento más interno.
\item Se ha adquirido una amplia soltura en el manejo de la herramienta de diseño FreeCad y en multitud de sus bancos de trabajo (Draft, A2Plus, Fasteners, Mesh...).
\item Se ha ganado una gran experiencia en el diseño de piezas mecánicas para su posterior impresión en 3D.
\item Aumento en la capacidad de planificación de tareas y organización de los recursos disponibles.
\item Manejo más rápido y descubrimiento de comandos nuevos de la herramienta Git.
\item Ampliación de conocimientos en ROS 2 y dominio pleno del formato URDF (particularidades, limitaciones y puntos fuertes).
\item Conocimiento del funcionamiento interno del framework MoveIt y la configuración de paquetes para usarse dentro de él.
\item Generar documentación de calidad para un trabajo y comprender la documentación de otros proyectos para poder 
utilizarlos posteriormente en proyectos propios.
\item Conocimientos avanzados en electrónica y en sistemas relacionados con el mundo de las máquinas CNC.
\item Gran dominio de los parámetros de impresión, para lograr piezas con las tolerancias correctas y un acabado visual perfecto. 
\end{itemize}

Por último, añade otro par de párrafos de líneas futuras; esto es, cómo se puede continuar tu trabajo para abarcar una solución más amplia, o qué otras ramas de la investigación podrían seguirse partiendo de este trabajo, o cómo se podría mejorar para conseguir una aplicación real de este desarrollo (si es que no se ha llegado a conseguir).
\section{Valoración final y líneas futuras}
Se ha desarrollado un robot eficaz y barato capaz de cumplir con su cometido de poder usarse en la docencia universitaria. Es por esto 
que se plantean las siguientes líneas futuras para continuar con el proyecto y mejorarlo todavía más:

\begin{itemize}
    \item Añadir un cuarto grado de libertad en el extremo del robot para poder rotar objetos en el plano de trabajo.
    \item Utilizar un rodamiento de mayor tamaño en la rotación de la base para reducir la flexión del robot al estar sometido 
    a cargas pesadas.
    \item Crear una serie de herramientas nuevas para acoplar a este robot como puede ser un porta rotulador, una pinza o un láser.
\end{itemize}


