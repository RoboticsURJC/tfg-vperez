\chapter{Conclusiones}
\label{cap:capitulo7}

\begin{flushright}
\begin{minipage}[]{10cm}
\emph{El genio se compone del 2\% de talento y del 98\% de perseverancia}\\
\end{minipage}\\

Beethoven\\
\end{flushright}

\vspace{1cm}

\noindent En este último capítulo se hará mención a los logros alcanzados en términos de objetivos y se presentarán 
las conclusiones obtenidas en este proyecto. También se discutirán las habilidades y conocimientos 
adquiridos durante su desarrollo, así como las posibles mejoras futuras a tener en cuenta.
\section{Objetivos cumplidos}
\noindent Primeramente cabe destacar que se ha logrado cumplir el objetivo principal de este trabajo: desarrollar un 
brazo robótico que pueda ser empleado en la asignatura de Robótica Industrial de esta universidad. De hecho el robot 
ha cumplido todos y cada uno de los requisitos planteados en el tercer capítulo:
\begin{enumerate}
\item El coste final de fabricación ha sido de 171\euro, siendo 200\euro \xspace el límite establecido.
\item En su totalidad está impreso en 3D, a excepción de los componentes electrónicos.
\item Se ha cumplido con mantener el consumo inferior a 25 vatios, siendo siempre inferior a 20.
\item El robot final es portable y cuenta con un tamaño ideal para usarse en una mesa normal. Además se ha logrado 
que pueda ser usado sin necesidad de anclarlse al suelo.
\item Es capaz de levantar los 300 gramos propuestos, incluso haciendo uso de la herramienta electromagnética desarrollada.
\item El robot creado es simple de montar y requiere de pocas piezas de tamaño medio que pueden ser impresas en cualquier 
impresora barata del mercado.
\item Se ha realizado la integración completa de este robot en el ecosistema ROS 2. Adicionalmente, se ha integrado en el 
framework MoveIt 2 para facilitar su uso.
\end{enumerate}

\section{Competencias adquiridas}
En el desarrollo de este trabajo se han adquirido una gran cantidad de conocimientos y competencias y conocimientos, 
entre los cuales destacan:
\begin{itemize}
\item Conocimientos avanzados acerca de Grbl y su funcionamiento más interno.
\item Se ha adquirido una amplia soltura en el manejo de la herramienta de diseño FreeCad y en multitud de sus bancos de trabajo (Draft, A2Plus, Fasteners, Mesh...).
\item Se ha ganado una gran experiencia en el diseño de piezas mecánicas para su posterior impresión en 3D.
\item Aumento en la capacidad de planificación de tareas y organización de los recursos disponibles.
\item Manejo más rápido y descubrimiento de comandos nuevos de la herramienta Git.
\item Ampliación de conocimientos en ROS 2 y dominio pleno del formato URDF (particularidades, limitaciones y puntos fuertes).
\item Conocimiento del funcionamiento interno del framework MoveIt y la configuración de paquetes para usarse dentro de él.
\item Generar documentación de calidad para un trabajo y comprender la documentación de otros proyectos para poder 
utilizarlos posteriormente en proyectos propios.
\item Conocimientos avanzados en electrónica y en sistemas relacionados con el mundo de las máquinas CNC.
\item Gran dominio de los parámetros de impresión, para lograr piezas con las tolerancias correctas y un acabado visual perfecto. 
\end{itemize}


\section{Valoración final y líneas futuras}
Se ha desarrollado un robot eficaz y barato capaz de cumplir con su cometido de poder usarse en la docencia universitaria. Es por esto 
que se plantean las siguientes líneas futuras para continuar con el proyecto y mejorarlo todavía más:

\begin{itemize}
    \item Añadir un cuarto grado de libertad en el extremo del robot para poder rotar objetos en el plano de trabajo.
    \item Utilizar un rodamiento de mayor tamaño en la rotación de la base para reducir la flexión del robot al estar sometido 
    a cargas pesadas.
    \item Crear una serie de herramientas nuevas para acoplar a este robot como puede ser un porta rotulador, una pinza o un láser.
    \item Integrar este robot en la asignatura de Robótica Industrial como se plantea en el Anexo II. 
\end{itemize}


