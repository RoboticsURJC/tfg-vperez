\chapter*{Anexo I}
\label{cap:anexoi}

\noindent En este anexo se indica paso a paso todo lo necesario para poder reproducir este robot a partir del contenido 
existente en el repositorio\footnote{\url{https://github.com/RoboticsURJC/tfg-vperez.git}} de este proyecto.

\section*{Obtención de los ficheros STL}
\noindent Para imprimir las distintas piezas, primeramente se requiere obtener los ficheros STL. Para ello, 
debemos utilizar los ficheros fuente (piezas de FreeCAD) para generarlos. Estos ficheros pueden 
encontrarse en la carpeta /src/design/FreeCad del proyecto o en este enlace\footnote{\url{https://github.com/RoboticsURJC/tfg-vperez/tree/9f9e47e46f0b0357938d444bf9baf661f90c402d/src/design/FreeCad}}.
\begin{enumerate}
\item Abrimos la pieza correspondiente con FreeCAD pulsando encima de ella.
\item Cambiamos al banco de trabajo \textit{Mesh Design}.
\item Pulsamos sobre la pieza para seleccionarla y posteriormente en el botón \textit{Crear malla de forma} del panel superior.
\item Introducimos un valor de superficie de desviación de 0.01mm y damos a OK.
\item Pulsamos el botón derecho del ratón sobre el objeto de malla creado y exportamos la malla con el botón \textit{Exportar malla}. 
    
\end{enumerate}

Una vez obtenidos todos los ficheros STL, se pueden importar en el laminador, en este caso se ha empleado 
Ultimaker Cura 5.2.1 para generar los ficheros que lee la impresora. La densidad requerida para cada pieza 
y la cantidad de ellas, se pueden ver en el Cuadro \ref{cuadro:piezas} del Capítulo 5.

\newpage
\section*{Montaje}
\noindent Actualmente, no se dispone de una guía detallada de montaje. Aún así el montaje resulta bastante sencillo 
e intuitivo. La manera recomendada de proceder es fijándose en el fichero de 
montaje\footnote{\url{https://github.com/RoboticsURJC/tfg-vperez/blob/280861172bce3b1c0cfbb155a434364ea68eeb30/src/design/FreeCad/\%230\_ASSEMBLY.FCStd}}. En él 
aparece cada pieza, su posición y nombre. En la Sección \ref{sec:impresionymontaje} se puede ver la lista de las piezas que 
son necesarias para poder comprarlas o fabricarlas.

\section*{Software}
\noindent Todo el software necesario para hacer funcionar este robot se encuentra en la carpeta /src/software del repositorio. Dentro de 
esta carpeta existen una serie de paquetes para utilizar el robot, los cuales son:
\begin{itemize}
\item Paquete \textit{g\_arm}: Es un paquete de ROS 2 que contiene el software necesario para comunicarse con el robot desde el 
ecosistema ROS. Dentro de él se encuentra la carpeta \textit{g\_arm\_lib} con las clases en python necesarias para abstraer al 
ejecutable de ROS de las comunicaciones.
\item Paquete \textit{g\_arm\_description}: Es un paquete de descripción creado para representar el robot real en el mundo virtual. No contiene código, únicamente una serie de lanzadores para poder visualizarlo.
\item Paquete \textit{g\_arm\_moveit2}: Es un paquete de MoveIt 2 que utiliza el anterior paquete para poder controlar el robot 
real desde este framework.
\end{itemize}

Aunque se puede controlar las articulaciones del robot a bajo nivel a base de enviar órdenes de código G a través del puerto serie, se pierde la 
posibilidad de conocer la posición absoluta 
de cada articulación. Debido a esto, es recomendable utilizar al menos la capa de abstracción \textit{Robot} que se puede encontrar en 
la carpeta \textit{g\_arm\_lib} del paquete \textit{g\_arm}. Para ver como se utiliza, se puede ver el código del ejecutable 
\textit{driver.py}. 

Aún así, lo ideal es utilizar ROS 2 Humble y MoveIt 2 para controlar el robot. Para ello, es necesario tener instalado 
ambos (en internet existen guías paso a paso) y hacer uso de los tres paquetes incluidos en la carpeta /src/software del repositorio. 
Para utilizar estos paquetes es necesario añadirlos a una carpeta \textit{workspace} 

Para el firmware del robot tal
