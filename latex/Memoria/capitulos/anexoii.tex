\chapter*{Anexo II}
\label{cap:anexoii}

\noindent En este anexo se aborda el cómo se pretende incluir este robot en la asignatura de Robótica Industrial. Para ello 
se ha preguntado a Julio Lora, actual profesor de esta asignatura, cómo podría integrarse este trabajo en el siguiente curso académico. La respuesta 
fue que una de las maneras es aprender a generar trayectorias de soldadura mediante código, a través del framework de Move It. 


Para esta tarea se ha optado por emplear el lenguaje de programación Python, por su poca complejidad y similitud a MatLab. Ya que se ha optado por 
este lenguaje, se ha utilizado la librería PyMoveIt2\footnote{\url{https://github.com/AndrejOrsula/pymoveit2}} que proporciona una serie de métodos y herramientas que 
facilitan la interación con el robot. Para instalarla sólamente hace falta seguir los pasos que se indican en el enlace. Esta librería permite controlar 
el robot de diferentes formas:
\begin{enumerate}
\item \textit{Joint goal}: permite controlar el robot en el espacio de articulaciones, es decir llevar cada articulación del robot a posición angular concreta.
\item \textit{Pose goal}: permite controlar el robot en el espacio cartesiano, es decir llevar al extremo del robot a una posición XYZ con una 
cierta orientación.
\item \textit{Gripper action}: permite controlar la herramienta del robot a través de una serie de cómodas funciones.
\item \textit{Servo}: permite mandar comandos en tiempo real (sin la previa planificación que requieren los anteriores) para controlar las velocidad lineares y 
angulares del extremo del robot.

\end{enumerate}

Para realizar las trayectorias es necesario utilizar el submódulo \textit{Pose goal}. Debido a la geometría de este robot, sólamente se va a controlar la posición 
del extremo en el espacio. La forma de trazar una trayectoria es como una secuencia de puntos próximos entre sí. Primeramente, se ha realizado un 
programa de ejemplo\footnote{\url{https://github.com/RoboticsURJC/tfg-vperez/blob/main/src/software/g\_arm\_python_examples/g\_arm_python\_examples/goal.py}} 
cuyo único propósito es mover el robot a una posición concreta para probar su funcionamiento. Este ejemplo se encuentra en el paquete
 \texttit{g_arm_python_examples} del repositorio de este trabajo y puede ejecutarse. Es importante recalcar que si el punto no se encuentra dentro del 
espacio de trabajo el plan falla y no se ejecuta el movimiento (un error en rojo aparecerá en la terminal).


Para demostrar su uso en trayectorias, se han realizado 3 figuras geométricas parametrizadas por código para dibujar en un papel en el plano XY. Para acoplar 
un rotulador al robot real se ha diseñado una herramienta que encaja en el acople para herramientas de su extremo.


Podemos hacer este ejemplo lanzando:
ros2 launch g\_arm\_moveit2 demo.launch
ros2 run g\_arm\_python\_examples drawFigures