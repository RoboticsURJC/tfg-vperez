\cleardoublepage

\chapter*{Resumen\markboth{Resumen}{Resumen}}
La robótica educativa, principal pilar de la enseñanza moderna, 
se caracteriza por ofrecer robots útiles para el aprendizaje. Dentro de dicha rama
encontramos la robótica de bajo coste, campo de estudio que busca superar las barreras 
económicas que presentan los robots tradicionales para garantizar que todo el mundo 
pueda acceder a esta tecnología. 

Los robots usados comúnmente en las Universidades son realmente costosos por lo que 
es complicado adquirir una gran cantidad de ellos, limitando su disponibilidad. Es por esto que 
en este trabajo se ha desarrollado una solución para este problema. Concretamente, se ha desarrollado 
un brazo robótico industrial de muy bajo coste que puede ser fabricado mediante cualquier impresora 3D convencional.

Este proyecto se ha llevado a cabo mediante herramientas de diseño por ordenador, como es el caso de FreeCAD, una 
poderosa aplicación de modelado 3D de código abierto que ha permitido realizar cada una de las piezas que lo componen.

El hardware utilizado en este proyecto es fácilmente adquirible a través de internet, lo que ha permitido 
desarrollar el robot de manera accesible y económica. La elección de componentes baratos y de calidad disponibles 
en el mercado ha sido fundamental para garantizar su éxito.

El robot final ha sido integrado en ROS 2 Humble, un \textit{middleware} de código abierto ampliamente 
utilizado en aplicaciones robóticas. Esta integración ha proporcionado una interfaz 
estandarizada y eficiente para que cualquier aplicación desarrollada en este ecosistema pueda hacer uso de él. Además, 
ha sido configurado para utilizarse en el \textit{framework} MoveIt 2, lo que facilita aún más su uso.

Con la combinación de estas herramientas y tecnologías, se ha logrado desarrollar un robot altamente 
funcional y resistente, preparado para utilizarse en un entorno académico formando a nuevas generaciones. El 
resultado obtenido representa un esfuerzo de investigación y desarrollo significativo, que abre la puerta a futuras 
mejoras y avances en el campo de la robótica.

Finalmente, se han realizado numerosas pruebas que evalúan los distintos aspectos técnicos que 
definen el rendimiento y características del robot final.